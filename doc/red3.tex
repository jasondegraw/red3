\documentclass[10pt]{article}
\usepackage{amsmath}
\usepackage{mathptmx}
\usepackage{graphicx,color}
\usepackage[margin=1in]{geometry}
\usepackage[all]{xy}
\usepackage{nameref}
\usepackage[colorlinks=true,linkcolor=linkblue]{hyperref}
\definecolor{linkblue}{rgb}{0,0,1}
%
% Local macros
%
\newcommand{\eqn}[1]{Equation~\ref{#1}}
\newcommand{\reynum}{\ensuremath{\operatorname{Re}}}
\newcommand{\NS}{\hbox{Navier}-\hbox{Stokes}\ }
%
% Linking macros - all the autorefs/namerefs below need to be *'ed (as
%                  soon as I can use the newest hyperref version)
%
\newcommand{\batref}[1]{
\hyperref[#1]{\autoref{#1}}}
%
\newcommand{\batrefp}[1]{
\hyperref[#1]{\autoref{#1} 
(on page~\pageref*{#1})}}
%
\newcommand{\batrefn}[1]{
\hyperref[#1]{\autoref{#1} 
(\nameref{#1})}}
\newcommand{\batrefnp}[1]{
\hyperref[#1]{\autoref{#1} 
(\nameref{#1}, on page~\pageref*{#1})}}
%
\newcommand{\batrefo}[1]{
\hyperref[#1]{See section~\ref*{#1} 
(on page~\pageref*{#1})}}
%
% Change section to Section
%
\renewcommand{\sectionautorefname}{Section}
%
%
%
\title{RED3 Manual}
\author{Jason W. DeGraw}
\date{\today}
\begin{document}
\maketitle
\tableofcontents
%
\section{Introduction}
RED3 is an implementation in C++ of the staggered-grid fluid simulation
methodology.  It owes a great deal to the approximately factored
approach of Kim and Moin \cite{kim_moin85}.
%
\section{Governing Equations}
This section outlines the governing equations used in the solution process.
\subsection{Conservation of Mass}
The general expression for differential conservation of mass is
\begin{equation}\label{general_continuity}
\frac{\partial\rho}{\partial t} + \frac{\partial}{\partial x_j}
\left(\rho u_j\right) = 0.
\end{equation}
If the density $\rho$ is constant or approximately constant, then the flow is
termed incompressible and \eqn{general_continuity} reduces to
\begin{equation}\label{incompressible_continuity}
\frac{\partial u_j}{\partial x_j}=0.
\end{equation}
%
\subsection{Conservation of Momentum}
\begin{equation}\label{general_navier_stokes}
\frac{\partial}{\partial t}\left(\rho u_i\right)  + \frac{\partial}{\partial x_j}
\left(\rho u_ju_i\right) = -\frac{\partial p}{\partial x_i}
 + \frac{\partial}{\partial x_j}\left(\mu\frac{\partial u_i}{\partial x_j}\right)
 + f_i.
\end{equation}
%
\section{Approximate Factorization Method}\label{fracstep} The
numerical solution method used for the LES is an approximately
factored finite difference method developed by Kim and Moin
\cite{kim_moin85} in which the flow field is represented on a
staggered grid.  The method preserves conservation of momentum,
energy, and mass to machine precision.  The viscous terms are
spatially second-order central differenced.  The second-order temporal
treatment is Adams-Bashforth for the convective term and
Crank-Nicolson for the viscous term.  This method has been
successfully used for various flows by many researchers.

The incompressible \NS equations in cartesian coordinates are
\begin{equation}\label{iNS}
\frac{\partial u_i}{\partial t} + \frac{\partial}{\partial x_j}
\left(u_i u_j\right) = -\frac{\partial p}{\partial x_i} 
+ \frac{1}{\reynum}\frac{\partial^2 u_i}{\partial x_j \partial x_j}
\end{equation}
The time derivative in the \NS equations is second-order central
differenced about $n+1/2$ as
\begin{equation}
\frac{\partial u_i}{\partial t}\approx \frac{u_i^{n+1}-u_i^n}{\Delta t},
\end{equation}
where the superscript denotes timestep and $\Delta t$ is the timestep
size.  We represent the convective term by first defining
\begin{equation}
H_i^n = -\delta_i (u_i^n u_j^n),
\end{equation}
where $\delta_i$ is the $i$ direction finite difference operator.  The
convective term is calculated explicitly using an Adams-Bashforth
method:
\begin{equation}
\frac{\partial}{\partial x_i}(u_i u_j) 
\approx -\frac{1}{2}(3H_i^n - H_i^{n-1}).
\end{equation}
The viscous term is approximated implicitly using the Crank-Nicolson
method:
\begin{equation}
\frac{1}{Re}\frac{\partial^2 u_i}{\partial x_j \partial x_j}\approx
\frac{1}{2}\frac{1}{Re}(\delta_1^2+\delta_2^2+\delta_3^2)(u_i^{n+1}+u_i^n).
\end{equation}
Substituting these expressions into \autoref{iNS}, we have the
approximation
\begin{equation}\label{sdNS}
\frac{u_i^{n+1}-u_i^n}{\Delta t}= \frac{1}{2}(3H_i^n - H_i^{n-1})
	-\frac{\partial p}{\partial x_i}+
  \frac{1}{2}\frac{1}{Re}\left(\delta_1^2+\delta_2^2+\delta_3^2\right)
  \left(u_i^{n+1}+u_i^n\right).
\end{equation}
The velocity at the next timestep is decomposed into two components:
\begin{equation}\label{afdecomp}
u_i^{n+1}=\hat{u}_i + \Delta u_i.
\end{equation}
Substituting \autoref{afdecomp} into \autoref{sdNS} we obtain two
equations:
\begin{equation}\label{uhat}
\frac{\hat{u}_i-u_i^n}{\Delta t} = \frac{1}{2}(3H_i^n - H_i^{n-1})
  +\frac{1}{2}\frac{1}{Re}\left(\delta_1^2+\delta_2^2+\delta_3^2\right)
  \left(\hat{u}_i+u_i^n\right)
\end{equation}
and
\begin{equation}\label{Deltau}
\frac{\Delta u_i}{\Delta t}=-\frac{\partial p}{\partial x_i}+
  \frac{1}{2}\frac{1}{Re}\left(\delta_1^2+\delta_2^2+\delta_3^2\right)
  \Delta u_i.
\end{equation}
Splitting the equations in this manner defines the decomposition given
in \autoref{afdecomp}.  This is not the only such decomposition, but
this one is desirable because it constrains $\hat{u}_i$ to satisfy a
momentum-like equation that does not involve pressure.
\autoref{uhat} is rewritten as
\begin{equation}
(1-A_1-A_2-A_3)(\hat{u}_i-u_i^n)=\frac{\Delta t}{2}(3H_i^n - H_i^{n-1})
+2(A_1+A_2+A_3)u_i^n,
\end{equation}
where $A_i=(\Delta t / 2)\delta_i^2$.  Examining the left hand side,
we see that
\begin{equation}
1-A_1-A_2-A_3=(1-A_1)(1-A_2)(1-A_3)-A_1A_2-A_1A_3-A_2A_3+A_1A_2A_3.
\end{equation}
Neglecting all products of $A_i$'s, we have the approximately factored
momentum-like equation
\begin{equation}\label{afmom}
(1-A_1)(1-A_2)(1-A_3)(\hat{u}_i-u_i^n)=\frac{\Delta t}{2}(3H_i^n - H_i^{n-1})
+2(A_1+A_2+A_3)u_i^n.
\end{equation}
\autoref{afmom} is easier to solve than \autoref{uhat} since
it only requires the inversion of tridiagonal matrices.

We now define a pressure-like scalar function $\phi$ such that 
\begin{equation}\label{phidef}
\frac{\Delta u_i}{\Delta t}=\frac{\partial \phi}{\partial x_i}.
\end{equation}
Since $u_i^{n+1}$ must satisfy the continuity equation, we have
\begin{equation}\label{decompdiv}
\frac{\partial \hat{u}_j}{\partial x_j}
=-\frac{\partial}{\partial x_j}\Delta u_j.
\end{equation}
Combining Equations~\ref{phidef} and \ref{decompdiv} we have
\begin{equation}\label{phipoisson}
\frac{\partial \hat{u}_j}{\partial x_j}=\Delta t\frac{\partial^2 \phi}{\partial x_j \partial x_j}.
\end{equation}
This equation is a Poisson equation for $\phi$ which may may be solved
by a variety of methods.  We use a Fourier cosine transform method.
We solve for $\phi$ of form
\begin{equation}
\phi(i,j,k)=\sum^{N_x-1}_{l=0}\ \sum^{N_z-1}_{m=0} \tilde{\phi}(l,j,m)
\cos \left[\frac{\pi l}{N_x}\left(i+\frac{1}{2}\right)\right]
\cos \left[\frac{\pi m}{N_z}\left(k+\frac{1}{2}\right)\right],
\end{equation}
where $i=0,\dots ,N_x-1$, $j=0,\dots ,N_y-1$, $k=0,\dots ,N_z-1$, and $l$
and $m$ are the wave numbers.  Substituting this into
\autoref{phipoisson}, we obtain
\begin{equation}\label{phieq}
\delta_2^2 \tilde{\phi} - k'_l\tilde{\phi} -
k'_m\tilde{\phi}=\tilde{Q}(l,j,m),
\end{equation}
where $\tilde{Q}$ is the Fourier transform of $Q=\Delta
t\frac{\partial^2 \phi}{\partial x_j \partial x_j}$, and $k'_l$ and
$k'_m$ are the modified wave numbers defined as
\begin{equation}
k'_l=\frac{2}{\Delta x^2}\cos \left(\frac{\pi l}{N_x}\right)
\hbox{\ \ and\ \ }
k'_l=\frac{2}{\Delta z^2}\cos \left(\frac{\pi m}{N_z}\right).
\end{equation}
This equation results in a tridiagonal system of equations, which may
be easily solved. We need only perform an inverse Fourier transform to
find $\phi$.  

We are now able to advance the solution process in time.
%(provided that boundary conditions are correctly applied , see
%Section~\ref{bc}).  
Given $u_i^n$, we solve \autoref{afmom} for $\hat{u}_i$.  We then
may solve \autoref{phieq} for $\phi$, which in turn allows us to
find $\Delta u_i$ via \autoref{phidef}.  Then we have $u_i^{n+1}$
(\autoref{afdecomp}).  To compute the pressure, which is not
required for this algorithm, we combine Equations~\ref{Deltau} and
\ref{phidef} to obtain (after one integration)
\begin{equation}
p=\phi-\frac{\Delta t}{2}\frac{1}{Re}
\left(\delta_1^2+\delta_2^2+\delta_3^2\right)\phi.
\end{equation}
%Thus, once we know $\phi$ it is easy to calculate $p$.

\section{Boundary Conditions}\label{bc}
For the velocities, we use periodic conditions in the $x$ and $z$
directions.  In the the $z$ direction, this is accomplished by
identifying the indices of the solution matrices with one another on
the $z$ boundaries.  The $x$ direction implementation is slightly more
complicated, as the current implementation requires a mass flux to
drive the flow.  The mass flux is established by the initial
condition, it must be maintained by the solution procedure.  This is
accomplished by computing the intermediate flow as if it were under
inflow/outflow conditions, with a specified inflow at $i=1$.  When we
impose the divergence-free condition, we copy the ``outflow'' over the
``inflow'' and account for any differences, thus maintaining mass
flux.

%We also must specify boundary conditions for the intermediate velocity
%$\hat{u}_i$.  Appropriate conditions may be generated by considering
%$\hat{u}_i$ to be an approximation to $u_i^{\star}$, which we define
%to be a continuous function such that
%\begin{eqnarray}
%\frac{\partial u_i^{\star}}{\partial t} &=&H_i^\star+\frac{1}{Re}
%\frac{\partial}{\partial x_j}\frac{\partial}{\partial x_j}u_i^\star\\
%\noalign{\hbox{that agrees with $u_i$ at each timestep}\nonumber}
%u_i^{\star}(x_i,n\Delta t)&=&u_i(x_i,n\Delta t),
%\end{eqnarray}
%where $n$ is the number of timesteps taken and $\Delta t$ is the
%timestep size.  Then, expanding $u_i^{\star}$ as a Taylor series in
%time at the $(n+1)$st step:
%\begin{eqnarray*}
%\hat{u}_i &\approx& u_i^{\star}(x_i,[n+1]\Delta t)\\
%&=&u_i^{\star}(x_i,n\Delta t)+\Delta t\frac{\partial
%u_i^{\star}}{\partial t}
%+\frac{1}{2}\Delta t^2 \frac{\partial^2 u _i^{\star}}{\partial t^2}+\dots\\
%&=&u_i^{\star}(x_i,n\Delta t)+\Delta t\left(H_i^\star+\frac{1}{Re}
%\frac{\partial}{\partial x_j}\frac{\partial}{\partial
%x_j}u_i^\star\right)
%+\dots
%\end{eqnarray*}
%Since $u_i^{\star}(x_i,n\Delta t)=u_i(x_i,n\Delta t)$ and a
%rearrangement of the \NS equations gives
%\begin{equation}
%\frac{\partial u_i}{\partial t}+\frac{\partial p}{\partial x_i}=
%H_i+\frac{1}{Re}
%\frac{\partial}{\partial x_j}\frac{\partial}{\partial x_j}u_i,
%\end{equation}
%$\hat{u}_i$ is no given by 
%\begin{eqnarray}
%\hat{u}_i &\approx&u_i(x_i,n\Delta t)+\Delta t\left(\frac{\partial u_i}
%{\partial t}+\frac{\partial p}{\partial x_i}\right)
%+\dots\nonumber\\
%\hat{u}_i &\approx&u_i(x_i,[n+1]\Delta t)+\Delta t\frac{\partial
%p}{\partial x_i}+ \label{uhatbc}\dots
%\end{eqnarray}
%Keeping the first two terms of Equation~\ref{uhatbc} gives us an
%appropriate boundary condition for $\hat{u}_i$.

\section{Computational Grid}\label{grid}
%
%\staggridfig
% 
The numerical formulation employs a staggered grid, of which one grid cell is shown in Figure~\ref{staggrid}.  The physical locations where quantities are calculated are given in Table~\ref{gridtbl}.  
%Note that in Figure~\ref{staggrid} the pressure node is at the center
%of the cell.  
%\renewcommand\arraystretch{2}
\begin{table}[p]
\begin{center}
\caption{Node Locations.}
\label{gridtbl}
\begin{tabular}{|c|c|}\hline
\textit{Quantity} & \textit{Location}\\ \hline %&\textit{Index Range}\\ \hline 
$u(i,j,k)$  & $\left((x(i),y_m(j),z(k)\right)$   \\ \hline
%& $i=1\dots i_{max}$, $j=1\dots j_{max}+1$, and $k=1\dots k_{max}$ \\ \hline
$v(i,j,k)$  & $\left(x(i)-\frac{\Delta_1}{2},y(j), z(k)\right)$ \\ \hline
%& $i=1\dots i_{max}$, $j=1\dots j_{max}$, and $k=1\dots k_{max}$ \\ \hline
$w(i,j,k)$&$\left((x(i)-\frac{\Delta_1}{2},y_m(j),z(k)+\frac{\Delta_3}{2}\right)$ \\ \hline
%& $i=1\dots i_{max}$, $j=1\dots j_{max}+1$, and $k=1\dots k_{max}$ \\ \hline
$p(i,j,k)$  & $\left(x(i)-\frac{\Delta_1}{2},y_m(j),z(k)+\frac{\Delta_3}{2}\right)$ \\ \hline
\end{tabular}
\end{center}
\end{table}
%\label{gridsection}
%\input{grid}
In all simulations, we use uniform grid spacing in the $x$ and $z$
directions, with grid spacings $\Delta_1$ and $\Delta_3$,
respectively.  In the $y$ direction, we use a stretched grid created
using the hyperbolic tangent approach described in
Section~\ref{gridsection}.  The $y$ location of the $j$th $v$ velocity
point is denoted by $y(j)$.  The $u$ and $w$ points are located midway
between successive $v$ gridpoints, so we denote their $y$ locations by
$y_m(j)$:
\begin{equation}
y_m(j)=\frac{y(j)+y(j-1)}{2}\ \ \hbox{ for }j\in[2,\hbox{jmax}],
\end{equation}
where jmax is the number of $v$ gridpoints.  For convenience,
$y_m(1)=-1$.  We first determine the $y$ direction grid for $v$ using
\autoref{gridtanh}, and then the $y$ direction grid for $u$ (and
$w$) is calculated.
%
%The $v$ grid is defined by
%\begin{equation}
%y(j)=\frac{\tanh (\xi_j \theta )}{\tanh (\theta )},
%\end{equation}
%where $\theta$ is a constant and $\xi_j$ is defined as
%\begin{equation}
%\xi_j=\frac{j-N_h}{N_h-1},
%\end{equation}

\section{Differencing Formulas}
An important part of the implementation is the definition of the
proper differencing formulas.  The basic derivative approach is to
build higher order derivatives out of lower order ones:
\begin{equation}
\delta_i^2 \psi = \delta_i\left(\delta_i \psi\right)
\end{equation}
In terms of the discretization depicted in \autoref{diffschem}, this
is written as
\begin{equation}
\delta_i^2 \psi = \frac{\delta_i \psi|_{O^+}-\delta_i
\psi|_{\text{O}^-}}{d_3} 
= \frac{\psi_\text{E}-\psi_\text{O}}{d_1d_3}  
 -\frac{\psi_\text{O}-\psi_\text{W}}{d_2d_3}
= \frac{1}{d_1d_3}\psi_\text{E} - \frac{d_1+d_2}{d_1d_2d_3}\psi_\text{O}
 +\frac{1}{d_2d_3}\psi_\text{W}
\end{equation}\
For uniform spacings, this reduces to the standard second-order
difference formula.

\begin{figure}
%\begin{center}
%\begin{xy}
\[
\xy
(0,0)*{}; (40,0)*{} **\dir{-};
(0,0)*{\bullet}+(0,2)*{\psi_\text{W}};
(20,0)*{\bullet}+(0,2)*{\psi_\text{O}};
(40,0)*{\bullet}+(0,2)*{\psi_\text{E}};
(9,-1)*{}; (11,1)*{} **\dir{-};
(11,-1)*{}; (9,1)*{} **\dir{-};
(29,-1)*{}; (31,1)*{} **\dir{-};
(31,-1)*{}; (29,1)*{} **\dir{-};
(0,-2)*{}; (0,-12)*{} **\dir{-};
(20,-2)*{}; (20,-12)*{} **\dir{-};
{\ar@{<->}|{d_2} (0,-7)*{};(20,-7)*{}}; 
(40,-2)*{}; (40,-12)*{} **\dir{-};
{\ar@{<->}|{d_1} (20,-7)*{};(40,-7)*{}}; 
(10,2)*{}; (10,12)*{} **\dir{-};
(30,2)*{}; (30,12)*{} **\dir{-};
{\ar@{<->}|{d_3} (10,7)*{};(30,7)*{}};
(11,-3)*{\text{O}^-};
(31,-3)*{\text{O}^+};
\endxy
\]
%\end{xy}
%\end{center}
\caption{Differencing Schematic}\label{diffschem}
\end{figure}

\section{Storage Scheme}
The current version utilizes a single array approach for storing the
solution varibles.  All velocites are stored in a single array, as is
pressure.  The data structure is 
\begin{verbatim}
typedef struct {
  int nx;            /* Number of x-direction cells */
  int ny;            /* Number of y-direction cells */
  int nz;            /* Number of z-direction cells */
  real_t xl,yl,zl;   /* x-, y-, and z-direction dimensions (size)*/
  int dim;           /* Dimension of solution (2 or 3) */
  int ncell;         /* Total number of cells */
  int nu,nv,nw;      /* Number of xi-direction velocity entries */
  int nuvw;          /* Total number of velocity entries */
  real_t *x,*y,*z;   /* Grid spacings */
  real_t *u,*v,*w;   /* Velocity "arrays" - v and w are faked */
  real_t *p;         /* Pressure array */
  int iv,iw;         /* Shifts in the solution array to get v and w */
  int nsa;           /* Total length of solution array */
} strz_t;
\end{verbatim}

\section{Validation}\label{approxfactval}
To validate the method, we have performed a laminar channel simulation
at low Reynolds number.  In the case of fully developed laminar
channel flow, the \NS equations can be analytically solved, resulting
in a parabolic velocity profile.  The computational domain was
$8\times 2 \times 2$, with a $65\times 65 \times 32$ grid (uniformly
spaced in the $x$ and $z$ directions).  The $y$ direction is
determined as described in Section~\ref{lesgrid}, with the parameters
as in \autoref{lesgridtbl}.  We find that we are able to start from
varying initial conditions and the result is a nicely parabolic
profile. We present here a result that was started with a trapezoidal
$u$ profile everywhere (and $v=w=0$).  The simulation was run at
$Re=1000$ with a timestep of $\Delta t=0.02$.  After 400 time units,
the solution is a nearly parabolic profile, as shown in
Figure~\ref{lam}.
%\lamfig
%
\begin{table}[t]
\caption{Validation Grid Parameters.}\label{lesgridtbl}
\begin{center}
\begin{tabular}{|c|c|}\hline
Parameter  & Value\\ \hline
$y(2)$ & $-9.9965\times 10^{-1}$ \\ \hline 
$\theta$ & $-3.5644$\\ \hline
\end{tabular}
\end{center}
\end{table}


\section{LES Implementation}
%
The implementation of the LES model in our numerical scheme is
straightforward.  We use an implicit box filter, which uses the finite
difference grid as the filter.  We need not modify the underlying
numerical procedure and we only need to add an eddy viscosity
computation.  The $\tau_{ij}$ term is calculated and advanced in the
same manner that the convective term is advanced.

\section{Spectralish Possion Solution}

The solution of the Poisson equation on uniform grids is desired that
will be faster than the typical finite difference solution.  Such a
solution procedure can be developed using the Fourier transform.  We
will provide details in one dimension, and the extension to two and
three dimensions is straightforward.  For $\phi$ subject to the
one-dimensional Poisson equation
\begin{equation}\label{onedpoission}
\frac{\partial^2\phi}{\partial x^2} = f
\end{equation}
we look for solutions of form 
\begin{equation}\label{psiform}
\phi(i)=\sum^{N_x-1}_{l=0} \tilde{\phi}(l)\psi(i,l)
%\cos \left[\frac{\pi l}{N_x}\left(i+\frac{1}{2}\right)\right]
\end{equation}
for $i=0,1,\ldots,N_x-1$, where $\psi(i,l)$ are basis functions.  The
grid is uniform with
\begin{equation}
x(i)=\left(i+\frac{1}{2}\right)\Delta x
\end{equation}
which places each $\phi(i)$ at the center of an interval $\Delta x$ in
length.  Substituting \autoref{psiform} into \autoref{onedpoission},
multiplying by weight functions $w(i,m)$ and integrating over the
domain, we obtain the Galerkin weak form
\begin{equation}\label{weakform}
\int \left(\frac{\partial^2}{\partial x^2}\sum^{N_x-1}_{l=0} \tilde{\phi}(l)\psi(i,l) - 
           \sum^{N_x-1}_{l=0} \tilde{f}(l)\psi(i,l)\right)w(i,m)dx=0.
\end{equation}
Assuming that the basis functions and the weight functions are
orthogonal, and that the differentiation can be reinterpreted in terms
of a modified wavenumber expression $k^{**}$, \autoref{weakform} can
be rewritten as
\begin{equation}
-k^{**} \tilde{\phi}(l) = \tilde{f}(l).
\end{equation}
The minus sign is apparently some historical artifact, as it is not required.

\subsection{Neumann Boundary Conditions}
For Neumann boundary conditions, we choose as basis functions the
cosine:
\begin{equation}
\psi(i,l)=\cos \left[\frac{\pi l}{N_x}\left(i+\frac{1}{2}\right)\right]
\end{equation}
which leads to solutions of the form
\begin{equation}
\phi(i)=\sum^{N_x-1}_{l=0} \tilde{\phi}(l)
\cos \left[\frac{\pi l}{N_x}\left(i+\frac{1}{2}\right)\right]
\end{equation}
In this form, the coefficients $\tilde{\phi}(l)$ are the result of the
discrete cosine transform (more specifically, the DCT-II).  We must
determine the modified wavenumbers to proceed.  
\begin{equation}
\frac{\partial^2}{\partial x^2} 
\cos \left[\frac{\pi l}{N_x}\left(i+\frac{1}{2}\right)\right] \approx
\frac{ \cos \left[\frac{\pi l}{N_x}\left(i+\frac{1}{2}-1\right)\right]
- 2 \cos \left[\frac{\pi l}{N_x}\left(i+\frac{1}{2}\right)\right]
+ \cos \left[\frac{\pi l}{N_x}\left(i+\frac{1}{2}+1\right)\right]}
{\Delta x^2}
\end{equation}

 on the uniform which may may be solved by a variety of
methods.  We use a Fourier cosine transform method.  We solve for
$\phi$ of form
\begin{equation}
\phi(i,j,k)=\sum^{N_x-1}_{l=0}\ \sum^{N_z-1}_{m=0} \tilde{\phi}(l,j,m)
\cos \left[\frac{\pi l}{N_x}\left(i+\frac{1}{2}\right)\right]
\cos \left[\frac{\pi m}{N_z}\left(k+\frac{1}{2}\right)\right],
\end{equation}
where $i=0,\dots ,N_x-1$, $j=0,\dots ,N_y-1$, $k=0,\dots ,N_z-1$, and $l$
and $m$ are the wave numbers.  Substituting this into
\autoref{phipoisson}, we obtain
\begin{equation}\label{phieq}
\delta_2^2 \tilde{\phi} - k'_l\tilde{\phi} -
k'_m\tilde{\phi}=\tilde{Q}(l,j,m),
\end{equation}
where $\tilde{Q}$ is the Fourier transform of $Q=\Delta
t\frac{\partial^2 \phi}{\partial x_j \partial x_j}$, and $k'_l$ and
$k'_m$ are the modified wave numbers defined as
\begin{equation}
k'_l=\frac{2}{\Delta x^2}\cos \left(\frac{\pi l}{N_x}\right)
\hbox{\ \ and\ \ }
k'_m=\frac{2}{\Delta z^2}\cos \left(\frac{\pi m}{N_z}\right).
\end{equation}
This equation results in a tridiagonal system of equations, which may
be easily solved. We need only perform an inverse Fourier transform to
find $\phi$.  

\end{document}

